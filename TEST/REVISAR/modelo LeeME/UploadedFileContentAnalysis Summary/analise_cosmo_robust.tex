\documentclass[11pt,a4paper]{article}
\usepackage[utf8]{inputenc}
\usepackage[T1]{fontenc}
\usepackage[portuguese]{babel}
\usepackage{amsmath}
\usepackage{amssymb}
\usepackage{graphicx}
\usepackage[a4paper,margin=2.5cm]{geometry}
\usepackage{hyperref}
\usepackage{caption}
\usepackage{booktabs} % For better tables

\title{Análise Robusta do Modelo $\Lambda$CDM com Dados H(z) de Cosmic Chronometers}
\author{Análise realizada por Manus}
\date{\today}

\begin{document}

\maketitle

\begin{abstract}
O modelo $\Lambda$ Cold Dark Matter ($\Lambda$CDM) é o paradigma cosmológico padrão, mas enfrenta desafios como a Tensão de Hubble. Este trabalho reanalisa o modelo $\Lambda$CDM plano utilizando dados da função de Hubble H(z) obtidos a partir de Cosmic Chronometers (CC), focando em um tratamento rigoroso dos erros sistemáticos e estatísticos. Utilizamos um conjunto de 15 pontos de dados H(z) de Moresco et al. e a matriz de covariância completa (estatística + sistemática) para restringir os parâmetros cosmológicos através de uma análise de Cadeias de Markov Monte Carlo (MCMC). Obtivemos $H_0 = 62.99^{+11.23}_{-8.24}$ km s$^{-1}$ Mpc$^{-1}$ e $\Omega_m = 0.354^{+0.124}_{-0.095}$ (medianas e intervalos de 68\% C.L.). Estes resultados, embora com incertezas consideráveis, são consistentes com as medições do universo primordial (Planck/BAO) e destacam a importância crucial da matriz de covariância em análises cosmológicas com dados de CC.
\end{abstract}

\section{Introdução}
O modelo $\Lambda$CDM descreve com sucesso uma vasta gama de observações cosmológicas, desde a Radiação Cósmica de Fundo (CMB) até a distribuição de galáxias em grande escala. No entanto, uma tensão significativa, conhecida como Tensão de Hubble \cite{ref:tension_review}, surgiu entre as medições da taxa de expansão atual do Universo, $H_0$, inferidas a partir de observações do universo primordial (e.g., Planck Collaboration \cite{ref:planck2018}) e aquelas obtidas por métodos locais no universo tardio (e.g., Supernovas Tipo Ia \cite{ref:shoes}).

Neste contexto, sondas cosmológicas independentes são cruciais. Os Cosmic Chronometers (CC) oferecem um método para medir diretamente a função de Hubble $H(z) = -\frac{1}{1+z} \frac{dz}{dt}$ estudando a evolução da idade de populações estelares passivas em galáxias \cite{ref:cc_method}. Este método é independente da escala de distâncias cósmicas e pode fornecer restrições valiosas sobre $H_0$ e outros parâmetros.

Contudo, análises anteriores mostraram que as restrições cosmológicas derivadas de dados CC são altamente sensíveis ao tratamento dos erros sistemáticos e suas correlações. O objetivo deste trabalho é realizar uma análise robusta do modelo $\Lambda$CDM plano utilizando um conjunto atualizado de dados H(z) de CC (15 pontos de Moresco et al. \cite{ref:moresco_data}) e incorporando a matriz de covariância completa (estatística e sistemática) para obter restrições confiáveis nos parâmetros $H_0$ e $\Omega_m$.

\section{Dados e Metodologia}

\subsection{Dados H(z) de Cosmic Chronometers}
Utilizamos o conjunto de 15 medições de H(z) compiladas por Moresco et al. \cite{ref:moresco_data}, abrangendo o intervalo de redshift $0.1791 \le z \le 1.965$. Estes dados, juntamente com seus erros estatísticos ($\sigma_{stat}$), foram obtidos do arquivo \texttt{HzTable\_MM\_BC03.dat} disponível publicamente no repositório \texttt{CCcovariance} \cite{ref:cc_repo}. Os dados são apresentados na Tabela \ref{tab:hz_data}.

\begin{table}[htbp]
\centering
\caption{Dados H(z) de Cosmic Chronometers utilizados nesta análise \cite{ref:moresco_data}. A coluna $\sigma_{stat}$ representa apenas a componente estatística da incerteza.}
\label{tab:hz_data}
\begin{tabular}{ccc}
\toprule
Redshift (z) & H(z) [km s$^{-1}$ Mpc$^{-1}$] & $\sigma_{stat}$ [km s$^{-1}$ Mpc$^{-1}$] \\
\midrule
0.1791 & 74.91 & 3.8 \\
0.1993 & 74.96 & 4.9 \\
0.3519 & 82.78 & 13.0 \\
0.3802 & 83.00 & 4.3 \\
0.4004 & 76.97 & 2.1 \\
0.4247 & 87.08 & 2.4 \\
0.4497 & 92.78 & 4.5 \\
0.4783 & 80.91 & 2.1 \\
0.5929 & 103.80 & 11.6 \\
0.6797 & 91.60 & 6.4 \\
0.7812 & 104.50 & 9.4 \\
0.8754 & 125.10 & 15.3 \\
1.0370 & 153.70 & 13.6 \\
1.3630 & 160.00 & 23.07 \\
1.9650 & 186.50 & 35.05 \\
\bottomrule
\end{tabular}
\end{table}


\subsection{Matriz de Covariância}
A análise cosmológica rigorosa com dados CC requer a inclusão da matriz de covariância total, $\mathbf{C} = \mathbf{C}_{stat} + \mathbf{C}_{syst}$, que considera tanto os erros estatísticos quanto os sistemáticos e suas correlações.

A matriz de covariância estatística, $\mathbf{C}_{stat}$, é assumida como diagonal, com elementos $(\mathbf{C}_{stat})_{ii} = \sigma_{stat,i}^2$.

A matriz de covariância sistemática, $\mathbf{C}_{syst}$, captura as correlações induzidas por incertezas comuns na modelagem das populações estelares e outras fontes. Seguindo Moresco et al. \cite{ref:moresco_cov}, decompomos $\mathbf{C}_{syst}$ em contribuições da Função de Massa Inicial (IMF), biblioteca estelar (stlib), modelo de Síntese de População Estelar (SPS) e metalicidade (met). As componentes sistemáticas relativas foram obtidas do arquivo \texttt{hz\_data\_moresco2020.dat} \cite{ref:cc_repo}, interpoladas para os redshifts dos dados H(z), e combinadas com a componente diagonal da metalicidade (de \texttt{HzTable\_MM\_BC03.dat}) para construir $\mathbf{C}_{syst}$. A matriz de covariância total (15x15) resultante, salva como \texttt{cov\_matrix\_total\_hz.dat}, foi verificada como positiva definida.

\subsection{Modelo Cosmológico}
Assumimos um modelo $\Lambda$CDM espacialmente plano, onde a função de Hubble é dada por:
\begin{equation}
H(z) = H_0 \sqrt{\Omega_m (1+z)^3 + (1 - \Omega_m)}
\label{eq:Hz_lcdm}
\end{equation}
onde $H_0$ é a constante de Hubble e $\Omega_m$ é o parâmetro de densidade de matéria hoje.

\subsection{Análise Estatística}
Utilizamos uma análise Bayesiana empregando Cadeias de Markov Monte Carlo (MCMC) para explorar o espaço de parâmetros ($H_0$, $\Omega_m$). A função de log-verossimilhança ($\%\mathcal{L}$) é definida a partir do $\chi^2$ generalizado:
\begin{equation}
\ln \mathcal{L} \propto -\frac{1}{2} \chi^2 = -\frac{1}{2} \sum_{i,j} [H_{obs}(z_i) - H_{th}(z_i; H_0, \Omega_m)] (\mathbf{C}^{-1})_{ij} [H_{obs}(z_j) - H_{th}(z_j; H_0, \Omega_m)]
\label{eq:chi2}
\end{equation}
onde $H_{obs}$ são os dados observados, $H_{th}$ é o modelo teórico (Eq. \ref{eq:Hz_lcdm}), e $\mathbf{C}^{-1}$ é a inversa da matriz de covariância total.

Adotamos priors planos e largos para os parâmetros: $H_0 \in [50, 100]$ km s$^{-1}$ Mpc$^{-1}$ e $\Omega_m \in [0.01, 0.99]$. A implementação MCMC foi realizada com o pacote Python \texttt{emcee} \cite{ref:emcee}, utilizando 50 caminhantes (walkers) e 2000 passos, com 500 passos de burn-in descartados.

Calculamos também os critérios de informação de Akaike (AIC) e Bayesiano (BIC) para avaliar a qualidade do ajuste.

\section{Resultados}

A análise MCMC convergiu bem, e as distribuições posteriores para os parâmetros $H_0$ e $\Omega_m$ foram obtidas. Os valores medianos e os intervalos de confiança de 68\% (1$\sigma$) são apresentados na Tabela \ref{tab:mcmc_results}.

\begin{table}[htbp]
\centering
\caption{Resultados do ajuste MCMC para os parâmetros do modelo $\Lambda$CDM plano utilizando 15 pontos de dados H(z) de CC com matriz de covariância completa. Os valores reportados são as medianas das distribuições posteriores e os intervalos de confiança de 68\%.}
\label{tab:mcmc_results}
\begin{tabular}{lc}
\toprule
Parâmetro & Valor Ajustado \\
\midrule
$H_0$ [km s$^{-1}$ Mpc$^{-1}$] & $62.99^{+11.23}_{-8.24}$ \\
$\Omega_m$ & $0.354^{+0.124}_{-0.095}$ \\
\midrule
Estatística & Valor \\
\midrule
$\chi^2_{min}$ (nos parâmetros medianos) & 5.97 \\
Graus de Liberdade (dof) & 13 \\
$\chi^2_{min} / \text{dof}$ & 0.46 \\
AIC & 9.97 \\
BIC & 11.38 \\
\bottomrule
\end{tabular}
\end{table}

O gráfico de contorno (corner plot) na Figura \ref{fig:corner} visualiza as distribuições de probabilidade unidimensionais e bidimensionais para os parâmetros, mostrando também a correlação entre eles.

% Placeholder for Figure
\begin{figure}[htbp]
\centering
% Inserir o gráfico aqui. O arquivo deve estar no mesmo diretório ou caminho especificado.
\includegraphics[width=0.7\textwidth]{mcmc_corner_plot_hz_fullcov_corrected.png}
\caption{Distribuições de probabilidade posteriores para os parâmetros $H_0$ e $\Omega_m$ do modelo $\Lambda$CDM plano, ajustados aos 15 pontos de dados H(z) de CC com matriz de covariância completa. As linhas indicam as medianas e os intervalos de 68\% C.L.}
\label{fig:corner}
\end{figure}


\section{Discussão}
Os resultados obtidos com a análise robusta (N=15, covariância completa) diferem significativamente de uma análise preliminar que utilizou apenas 5 pontos de dados e erros estatísticos. Aquela análise inicial sugeriu $H_0 \approx 79$ km s$^{-1}$ Mpc$^{-1}$ e $\Omega_m \approx 0.11$, destacando a instabilidade de conclusões baseadas em dados limitados e tratamento inadequado de erros.

Nossa restrição em $H_0 = 62.99^{+11.23}_{-8.24}$ km s$^{-1}$ Mpc$^{-1}$ possui uma incerteza considerável, mas o valor central é notavelmente mais baixo que o valor local de SNe Ia ($H_0 \approx 73.4$ \cite{ref:shoes}) e mais próximo do valor inferido do CMB por Planck ($H_0 = 67.4 \pm 0.5$ \cite{ref:planck2018}). Dentro das grandes barras de erro, nosso resultado é consistente com Planck/BAO (<1$\sigma$ de diferença), mas mostra alguma tensão (>1$\sigma$) com o valor das SNe Ia. Portanto, estes dados CC, quando analisados rigorosamente, não favorecem fortemente um valor alto de $H_0$ para resolver a Tensão de Hubble.

O valor de $\Omega_m = 0.354^{+0.124}_{-0.095}$ é totalmente consistente com os valores derivados de Planck ($\Omega_m = 0.315 \pm 0.007$ \cite{ref:planck2018}) e outras sondas cosmológicas.

O baixo valor de $\chi^2 / \text{dof} \approx 0.46$ sugere que o modelo $\Lambda$CDM descreve bem os dados H(z) de CC, ou que as incertezas, particularmente as sistemáticas na compilação de Moresco et al., podem ser conservadoras (superestimadas).

\section{Conclusão}
Realizamos uma análise robusta do modelo $\Lambda$CDM plano utilizando 15 pontos de dados H(z) de Cosmic Chronometers e a matriz de covariância completa (estatística + sistemática). Obtivemos $H_0 = 62.99^{+11.23}_{-8.24}$ km s$^{-1}$ Mpc$^{-1}$ e $\Omega_m = 0.354^{+0.124}_{-0.095}$. Estes resultados, embora com incertezas relativamente grandes para $H_0$, são consistentes com o modelo cosmológico padrão e com as medições do universo primordial (Planck/BAO). A análise ressalta a importância crítica de utilizar conjuntos de dados adequados e um tratamento rigoroso dos erros e suas correlações para obter restrições cosmológicas confiáveis a partir de dados de Cosmic Chronometers.

\begin{thebibliography}{9}
% Placeholders - Citações reais devem ser adicionadas
\bibitem{ref:tension_review} Verde, L., Treu, T., & Riess, A. G. (2019). Tensions between the Early and the Late Universe. Nature Astronomy, 3(10), 891–895.
\bibitem{ref:planck2018} Planck Collaboration et al. (2020). Planck 2018 results. VI. Cosmological parameters. Astronomy & Astrophysics, 641, A6.
\bibitem{ref:shoes} Riess, A. G., et al. (2022). A Comprehensive Measurement of the Local Value of the Hubble Constant with 1 km/s/Mpc Uncertainty from the Hubble Space Telescope and the SH0ES Team. The Astrophysical Journal Letters, 934(1), L7.
\bibitem{ref:cc_method} Jimenez, R., & Loeb, A. (2002). Constraining Cosmological Parameters Based on Relative Galaxy Ages. The Astrophysical Journal, 573(1), 37–42.
\bibitem{ref:moresco_data} Moresco, M., et al. (2016). A 6\% measurement of the Hubble parameter at z~0.45: direct evidence of the epoch of cosmic re-acceleration. Journal of Cosmology and Astroparticle Physics, 2016(05), 014. (E dados subsequentes da colaboração)
\bibitem{ref:cc_repo} Moresco, M. GitLab Repository. \url{https://gitlab.com/mmoresco/CCcovariance}
\bibitem{ref:moresco_cov} Moresco, M., et al. (2020). Setting the stage for Cosmic Chronometers. II. Covariance matrix and prospects for upcoming surveys. Journal of Cosmology and Astroparticle Physics, 2020(08), 029.
\bibitem{ref:emcee} Foreman-Mackey, D., Hogg, D. W., Lang, D., & Goodman, J. (2013). emcee: The MCMC Hammer. Publications of the Astronomical Society of the Pacific, 125(925), 306–312.

\end{thebibliography}

\end{document}

