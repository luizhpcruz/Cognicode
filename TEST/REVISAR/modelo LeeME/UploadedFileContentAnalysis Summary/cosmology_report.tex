\documentclass{article}
\usepackage[utf8]{inputenc}
\usepackage{amsmath}
\usepackage{amsfonts}
\usepackage{amssymb}
\usepackage{graphicx}
\usepackage{float}
\usepackage{hyperref}
\usepackage{listings}
\usepackage[brazilian]{babel}

% Configurações para código Python
\lstset{
    language=Python,
    basicstyle=\ttfamily\small,
    commentstyle=\color{gray},
    keywordstyle=\color{blue},
    stringstyle=\color{red},
    showstringspaces=false,
    breaklines=true,
    frame=single,
    numbers=left,
    numberstyle=\tiny\color{gray},
    tabsize=4,
    captionpos=b,
    breakatwhitespace=false,
    breakindent=0pt,
    xleftmargin=0.5cm,
    xrightmargin=0.5cm,
    aboveskip=1em,
    belowskip=1em,
}

\title{Explorando Modelos Cosmológicos com Dados de Cosmic Chronometers: Uma Análise MCMC Robusta}
\author{Manus AI Agent}
\date{\today}

\begin{document}

\maketitle

\begin{abstract}
Este trabalho apresenta uma análise abrangente de modelos cosmológicos, focando no modelo $\Lambda$CDM padrão e sua extensão, o modelo wCDM, utilizando dados observacionais de H(z) de Cosmic Chronometers. A metodologia emprega simulações Markov Chain Monte Carlo (MCMC) com tratamento rigoroso da matriz de covariância total para inferir parâmetros cosmológicos como a Constante de Hubble ($H_0$), a densidade de matéria ($\Omega_m$) e o parâmetro da equação de estado da energia escura ($w$). Os resultados obtidos são comparados com a literatura e discutidos no contexto da Tensão de Hubble. O documento detalha as formulações teóricas, a implementação computacional e a interpretação dos resultados, servindo como um guia autoexplicativo para a análise cosmológica.
\end{abstract}

\section{Introdução}

A cosmologia moderna busca desvendar a composição e a evolução do nosso Universo. O modelo cosmológico padrão, conhecido como $\Lambda$CDM, tem sido notavelmente bem-sucedido em descrever uma vasta gama de fenômenos observacionais. No entanto, desafios persistentes, como a \href{https://en.wikipedia.org/wiki/Hubble_tension}{Tensão de Hubble} -- uma discrepância entre as medições da Constante de Hubble ($H_0$) obtidas por diferentes métodos -- impulsionam a investigação de modelos alternativos ou extensões do $\Lambda$CDM.

Este trabalho tem como hipótese central que a análise robusta de dados de H(z) de Cosmic Chronometers, incorporando uma matriz de covariância completa, pode fornecer restrições independentes e valiosas sobre os parâmetros cosmológicos, incluindo o parâmetro da equação de estado da energia escura ($w$), e contribuir para a compreensão da Tensão de Hubble. Exploraremos a consistência do modelo wCDM com esses dados, avaliando se uma energia escura dinâmica ($w \neq -1$) é favorecida pelas observações.

\section{Metodologia}

\subsection{Cosmic Chronometers}

Os Cosmic Chronometers (CC) são galáxias massivas e passivamente evoluindo que atuam como 


sondas cósmicas para medir diretamente a taxa de expansão do Universo, $H(z)$. O método baseia-se na premissa de que a diferença de idade (dt) entre duas populações estelares passivamente evoluindo em redshifts ligeiramente diferentes (dz) pode ser usada para inferir $H(z)$ através da relação:

$$ H(z) = -\frac{1}{1+z} \frac{dz}{dt} $$

A seleção de galáxias para atuar como CC segue critérios rigorosos para garantir a confiabilidade das medições:
\begin{itemize}
    \item \textbf{Galáxias Passivamente Evoluindo:} Ausência de formação estelar ativa, verificada por cortes fotométricos (e.g., NUVrJ) e espectroscópicos (ausência de linhas de emissão fortes).
    \item \textbf{Populações Estelares Antigas e Homogêneas:} Preferência por populações formadas em um único pulso para simplificar a modelagem da evolução da idade.
    \item \textbf{Massa Elevada:} Galáxias mais massivas tendem a ter histórias de formação estelar mais simples e rápidas.
    \item \textbf{Tratamento Rigoroso de Erros:} A inclusão de uma matriz de covariância completa é essencial para contabilizar as correlações entre as incertezas (estatísticas e sistemáticas).
\end{itemize}

\subsection{Equação de Hubble para Diferentes Modelos Cosmológicos}

A equação de Hubble, derivada das equações de Friedmann, descreve a taxa de expansão do Universo. Sua forma geral é:

$$ H^2(z) = H_0^2 [\Omega_m (1+z)^3 + \Omega_r (1+z)^4 + \Omega_k (1+z)^2 + \Omega_x f(z)] $$

Onde $H_0$ é a Constante de Hubble, $\Omega_m$, $\Omega_r$, $\Omega_k$ e $\Omega_x$ são os parâmetros de densidade de matéria, radiação, curvatura e energia escura, respectivamente, e $f(z)$ descreve a evolução da densidade da energia escura com o redshift.

\subsubsection{Modelo $\Lambda$CDM (Plano)}

No modelo $\Lambda$CDM plano, assume-se $\Omega_r \approx 0$, $\Omega_k = 0$, e a energia escura é uma constante cosmológica ($f(z)=1$). Além disso, $\Omega_m + \Omega_{\Lambda} = 1$. A equação de Hubble se simplifica para:

$$ H(z) = H_0 \sqrt{\Omega_m (1+z)^3 + (1-\Omega_m)} $$

\subsubsection{Modelo wCDM (Plano)}

O modelo wCDM estende o $\Lambda$CDM permitindo que a equação de estado da energia escura, $w$, seja um parâmetro livre e constante, mas não necessariamente igual a $-1$. Para um universo plano, a equação de Hubble é:

$$ H(z) = H_0 \sqrt{\Omega_m (1+z)^3 + (1-\Omega_m)(1+z)^{3(1+w)}} $$

\subsection{Simulações Markov Chain Monte Carlo (MCMC)}

Para inferir os parâmetros cosmológicos a partir dos dados de H(z), utilizamos simulações MCMC. Esta técnica permite explorar o espaço de parâmetros e construir as distribuições de probabilidade posteriores para cada parâmetro, dadas as observações e os priors. A função de log-verossimilhança é definida como:

$$ \ln \mathcal{L}(\mathbf{p} | \mathbf{D}) = -\frac{1}{2} (\mathbf{H}_{obs} - \mathbf{H}_{model}(\mathbf{p}))^T \mathbf{C}^{-1} (\mathbf{H}_{obs} - \mathbf{H}_{model}(\mathbf{p})) $$

Onde $\mathbf{p}$ são os parâmetros do modelo, $\mathbf{H}_{obs}$ são os valores observados de H(z), $\mathbf{H}_{model}$ são os valores preditos pelo modelo, e $\mathbf{C}^{-1}$ é a inversa da matriz de covariância total. Os priors são definidos como planos e largos para cada parâmetro, dentro de intervalos fisicamente razoáveis.

\section{Dados e Códigos Utilizados}

Os dados observacionais de H(z) foram extraídos do arquivo `HzTable_MM_BC03.dat`, que contém medições de Cosmic Chronometers. A matriz de covariância total, crucial para uma análise robusta, foi carregada do arquivo `cov_matrix_total_hz.dat`. Ambos os arquivos foram previamente analisados e compreendidos.

\subsection{Código Python para Simulação MCMC (wCDM)}

O script Python `wcdm_mcmc.py` implementa a simulação MCMC para o modelo wCDM. Abaixo está o código completo:

\lstinputlisting{/home/ubuntu/wcdm_mcmc.py}

\subsection{Código Python para Simulação MCMC ($\Lambda$CDM)}

Para fins de comparação, também utilizamos um script para o modelo $\Lambda$CDM plano, `run_mcmc_simulation.py`. Abaixo está o código completo:

\lstinputlisting{/home/ubuntu/run_mcmc_simulation.py}

\section{Resultados e Discussão}

As simulações MCMC foram executadas para ambos os modelos, $\Lambda$CDM e wCDM. Os resultados são resumidos abaixo.

\subsection{Resultados do Modelo $\Lambda$CDM}

Os resultados da simulação MCMC para o modelo $\Lambda$CDM plano, obtidos do arquivo `mcmc_results_summary.txt`, são:

\begin{itemize}
    \item \textbf{Constante de Hubble ($H_0$):} 59.48 $^{+12.58}_{-11.99}$ km/s/Mpc
    \item \textbf{Densidade de Matéria ($\Omega_m$):} 0.395 $^{+0.193}_{-0.120}$
\end{itemize}

Estes resultados indicam um valor de $H_0$ mais baixo e um $\Omega_m$ consistente com o consenso cosmológico, embora com incertezas consideráveis, refletindo a natureza dos dados de H(z) e a complexidade da análise.

\subsection{Resultados do Modelo wCDM}

Os resultados da simulação MCMC para o modelo wCDM, obtidos do arquivo `wcdm_mcmc_summary.txt`, são:

\begin{itemize}
    \item \textbf{Constante de Hubble ($H_0$):} 64.01 $^{+12.72}_{-9.16}$ km/s/Mpc
    \item \textbf{Densidade de Matéria ($\Omega_m$):} 0.348 $^{+0.124}_{-0.106}$
    \item \textbf{Parâmetro da Equação de Estado ($w$):} -1.311 $^{+0.600}_{-0.473}$
\end{itemize}

O corner plot gerado para o modelo wCDM (`wcdm_corner_plot.png`) visualiza as distribuições de probabilidade posteriores e as correlações entre os parâmetros $H_0$, $\Omega_m$ e $w$. A distribuição para $w$ mostra que, embora o valor central seja próximo de $-1$, há uma incerteza considerável, o que é esperado com o conjunto de dados atual.

\subsection{Comparação de Modelos (AIC/BIC)}

Para comparar os modelos $\Lambda$CDM e wCDM, utilizamos os critérios de informação de Akaike (AIC) e Bayesiano (BIC). Os valores obtidos para o modelo wCDM são:

\begin{itemize}
    \item \textbf{AIC:} 11.82
    \item \textbf{BIC:} 13.94
\end{itemize}

Comparando com os valores do modelo $\Lambda$CDM (que podem ser calculados a partir do `mcmc_results_summary.txt` ou `mcmc_fit_summary_hz_fullcov.txt`): 

*   Para o modelo $\Lambda$CDM, o `mcmc_fit_summary_hz_fullcov.txt` indica: AIC = 9.97 e BIC = 11.38.

Uma diferença de AIC ou BIC maior que 2 geralmente indica evidência positiva contra o modelo com maior valor, e maior que 10, evidência muito forte. Neste caso, o modelo $\Lambda$CDM (AIC=9.97, BIC=11.38) apresenta valores ligeiramente menores de AIC e BIC em comparação com o modelo wCDM (AIC=11.82, BIC=13.94). Isso sugere que, com os dados atuais, o modelo $\Lambda$CDM é marginalmente preferido em termos de ajuste e complexidade, ou que a inclusão de $w$ como um parâmetro livre não melhora significativamente o ajuste para justificar a complexidade adicional. No entanto, as incertezas são grandes e mais dados seriam necessários para uma conclusão definitiva.

\section{Conclusão}

Este trabalho demonstrou a aplicação de técnicas de inferência bayesiana (MCMC) para restringir parâmetros cosmológicos utilizando dados de H(z) de Cosmic Chronometers. A análise dos modelos $\Lambda$CDM e wCDM, com um tratamento cuidadoso da matriz de covariância, forneceu resultados consistentes com a literatura, embora com incertezas que refletem a quantidade e a qualidade dos dados. A hipótese de que a análise robusta de dados de H(z) pode fornecer restrições valiosas foi confirmada, e a metodologia apresentada serve como um guia para futuras investigações.

Embora o modelo $\Lambda$CDM continue sendo o modelo padrão, a exploração de extensões como o wCDM é crucial para testar os limites de nossa compreensão atual do Universo e investigar fenômenos como a Tensão de Hubble. O código e a metodologia apresentados aqui fornecem uma base sólida para futuras pesquisas, incluindo a incorporação de mais dados e a exploração de modelos cosmológicos mais complexos.

\bibliographystyle{plain}
\bibliography{references}

\end{document}

