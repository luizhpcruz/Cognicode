\documentclass[a4paper,12pt]{article}
\usepackage[utf8]{inputenc}
\usepackage[brazil]{babel}
\usepackage{amsmath, amssymb}
\usepackage{graphicx}
\usepackage{float}
\usepackage{booktabs}
\usepackage{caption}
\usepackage{geometry}
\usepackage{hyperref}
\usepackage{siunitx}
\usepackage{authblk}
\usepackage{fancyhdr}
\usepackage{datetime}

\geometry{margin=2.5cm}
\hypersetup{
    colorlinks=true,
    linkcolor=blue,
    citecolor=blue,
    urlcolor=blue
}
\pagestyle{fancy}
\fancyhf{}
\rhead{Análise Bayesiana de $H(z)$}
\lhead{Luiz}
\rfoot{\thepage}

\title{Análise Bayesiana de $H(z)$ com MCMC}
\author{Luiz}
\affil{Paraibuna, São Paulo, Brasil}
\date{\today}

\begin{document}

\maketitle

\begin{abstract}
Este relatório apresenta uma análise bayesiana dos dados observacionais da taxa de expansão do universo, $H(z)$, utilizando o método de Monte Carlo via Cadeias de Markov (MCMC). Os parâmetros cosmológicos estimados incluem o parâmetro de Hubble atual ($H_0$) e a densidade de matéria ($\Omega_m$). A análise foi realizada com o pacote \texttt{emcee} em Python, e os resultados são discutidos em detalhes.
\end{abstract}

\section{Introdução}

A constante de Hubble, $H_0$, e a densidade de matéria, $\Omega_m$, são parâmetros fundamentais na cosmologia moderna. Estimar esses parâmetros com precisão é crucial para compreender a dinâmica e a evolução do universo. Neste estudo, utilizamos dados observacionais de $H(z)$ para realizar uma análise bayesiana e estimar os valores desses parâmetros.

\section{Dados Observacionais}

Os dados utilizados na análise são apresentados na Tabela~\ref{tab:hz_data}.

\begin{table}[H]
\centering
\caption{Dados observacionais de $H(z)$.}
\label{tab:hz_data}
\begin{tabular}{cccc}
\toprule
Redshift ($z$) & $H_{\text{obs}}$ [km/s/Mpc] & $\sigma_{H}$ [km/s/Mpc] \\
\midrule
0.24 & 79.69 & 2.32 \\
0.35 & 84.40 & 1.90 \\
0.57 & 92.40 & 1.50 \\
0.73 & 97.30 & 1.20 \\
1.00 & 103.00 & 1.10 \\
\bottomrule
\end{tabular}
\end{table}

\section{Metodologia}

A análise foi realizada utilizando o método de Monte Carlo via Cadeias de Markov (MCMC) com o pacote \texttt{emcee} em Python. O modelo cosmológico adotado foi o $\Lambda$CDM, com a seguinte forma para $H(z)$:

\[
H(z) = H_0 \sqrt{\Omega_m (1 + z)^3 + (1 - \Omega_m)}
\]

Os priors adotados para os parâmetros foram:

\begin{itemize}
    \item $H_0$: distribuição normal com média 70 km/s/Mpc e desvio padrão 5 km/s/Mpc.
    \item $\Omega_m$: distribuição uniforme no intervalo [0, 1].
\end{itemize}

Foram utilizados 50 walkers e 5000 passos para cada cadeia, descartando os primeiros 1000 passos como burn-in.

\section{Resultados}

Os resultados obtidos para os parâmetros cosmológicos são apresentados na Tabela~\ref{tab:results}.

\begin{table}[H]
\centering
\caption{Estimativas dos parâmetros cosmológicos.}
\label{tab:results}
\begin{tabular}{ccc}
\toprule
Parâmetro & Valor Estimado & Intervalo de Credibilidade (68\%) \\
\midrule
$H_0$ [km/s/Mpc] & 79.01 & $^{+1.56}_{-1.54}$ \\
$\Omega_m$ & 0.107 & $^{+0.013}_{-0.012}$ \\
\bottomrule
\end{tabular}
\end{table}

\section{Discussão}

Os valores estimados para $H_0$ e $\Omega_m$ estão em concordância com estudos recentes, embora o valor de $H_0$ seja ligeiramente superior ao estimado pelo \textit{Planck}. A análise de convergência das cadeias MCMC foi realizada através da inspeção visual dos traços e do cálculo do tempo de autocorrelação integrado, indicando boa mistura e convergência das cadeias.

\section{Conclusão}

A análise bayesiana realizada forneceu estimativas consistentes para os parâmetros cosmológicos $H_0$ e $\Omega_m$. Futuras análises podem incluir modelos cosmológicos alternativos, como o wCDM, e a incorporação de dados adicionais para refinar as estimativas.

\section*{Agradecimentos}

Agradeço a todos que contribuíram para a realização deste estudo.

\end{document}
