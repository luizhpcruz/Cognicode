\documentclass[12pt]{article}
\usepackage[utf8]{inputenc}
\usepackage{amsmath, amssymb}
\usepackage{graphicx}
\usepackage{geometry}
\usepackage{booktabs}
\usepackage{hyperref}
\usepackage{caption}
\usepackage{subcaption}
\usepackage{siunitx}
\usepackage{authblk}

\geometry{a4paper, margin=1in}
\title{Análise Bayesiana do Parâmetro de Hubble $H(z)$ com MCMC}
\author{Seu Nome}
\affil{Instituição ou Afiliação}
\date{\today}

\begin{document}

\maketitle

\begin{abstract}
Este relatório apresenta uma análise bayesiana do parâmetro de Hubble $H(z)$ utilizando o método de Monte Carlo via Cadeias de Markov (MCMC). Os dados observacionais foram ajustados a um modelo cosmológico modificado, e os parâmetros foram estimados com suas respectivas incertezas.
\end{abstract}

\section{Introdução}

A determinação precisa do parâmetro de Hubble $H(z)$ é fundamental para a compreensão da expansão do universo. Neste estudo, aplicamos uma abordagem bayesiana para ajustar um modelo cosmológico modificado aos dados observacionais de $H(z)$.

\section{Metodologia}

Utilizamos o método MCMC para explorar o espaço de parâmetros do modelo. A função de verossimilhança foi definida com base na diferença entre os valores observados e modelados de $H(z)$, considerando as incertezas associadas.

\section{Resultados}

Os parâmetros estimados com seus respectivos intervalos de credibilidade são apresentados na Tabela~\ref{tab:params}.

\begin{table}[h!]
\centering
\caption{Parâmetros estimados do modelo cosmológico.}
\label{tab:params}
\begin{tabular}{lccc}
\toprule
Parâmetro & Valor Estimado & $+1\sigma$ & $-1\sigma$ \\
\midrule
$H_0$ [km/s/Mpc] & 78.19 & 1.23 & 1.84 \\
$\Omega_m$ & 0.110 & 0.011 & 0.007 \\
$R_0$ & 0.04587 & 0.03639 & 0.03190 \\
$n$ & -3.162 & 3.744 & 4.708 \\
\bottomrule
\end{tabular}
\end{table}

\begin{figure}[h!]
\centering
\includegraphics[width=0.8\textwidth]{corner_plot.png}
\caption{Distribuições posteriores dos parâmetros estimados.}
\label{fig:corner}
\end{figure}

\section{Conclusões}

A análise bayesiana permitiu estimar os parâmetros do modelo cosmológico com suas respectivas incertezas. Os resultados obtidos são consistentes com estudos anteriores e contribuem para a compreensão da dinâmica de expansão do universo.

\section*{Referências}

\begin{itemize}
    \item Yu, H., Ratra, B., \& Wang, F.-Y. (2018). Hubble parameter and baryon acoustic oscillation measurement constraints on the Hubble constant, the deviation from the spatially flat ΛCDM model, the deceleration–acceleration transition redshift, and spatial curvature. \textit{The Astrophysical Journal}, 856(1), 3.
    \item Jimenez, R., \& Loeb, A. (2002). Constraining cosmological parameters based on relative galaxy ages. \textit{The Astrophysical Journal}, 573(1), 37.
\end{itemize}

\end{document}
